\documentclass{article}
\usepackage{graphicx}
\usepackage{polski}
\usepackage[utf8]{inputenc}
\begin{document}

\title{Model Chandrasekhara / Smoluchowskiego  - 1 pudełko}
\author{Piotr Piękos}

\maketitle

\section{Oznaczenia}
W celach notacyjnych rozbijemy proces $X(t)$ (ilość żyjących osób) na dwa procesy: 
\begin{itemize}
\item$N(t)$ - Ilość narodzin
\item$S(t)$ - Ilość śmierci
\end{itemize}
Wtedy $X(t) = N(t) - S(t)$,
Dodatkowo oznaczymy intensywności procesów przez:
\begin{itemize}
\item$a_N$ - intensywność procesu narodzin
\item$a_S$ - parametr rozkładu wykładniczego odpowiadającego za długość życia
\end{itemize}
Dodatkowe oznaczenia:
\begin{itemize}
\item$I_X(t)$ - indeksy "żywych" zmiennych w momencie t.
\item$W_i$ - zmienna losowa (o rozkładzie wykladniczym z parametrem $a_S$) mówiąca o długości życia osoby $i$
\end{itemize}

Możnaby spróbować zamodelować $S(t)$ jako niejednorodny proces poissona z intensywnością zależną od $N(t)$. Ja jednak to rozdzieliłem jedynie ze względów notacyjnych.

\subsection{Prawa ewolucji}
$P(X(t+h) = x + 1 | X(t) = 1)$:

Korzystamy tutaj z faktu, że dla procesu Poissona (N) mamy: 
\begin{itemize}
\item $P(N(t+h) = n + 1 | N(t) = n) = a_N h + o(h)$
\item $P(N(t+h) \geq n + 2 | N(t) = n) = o(h)$
\end{itemize}
Dodatkowo:
\begin{itemize}
\item $P(N(t+h) = n | N(t) = n) = 1 - a_N h + o(h)$
\item $P(S(t+h) = s | S(t) = s, X(t) = x) = P(\forall_i \in I_X(t) W_i \geq h) + o(h) = \prod_{i \in I_X(t)} P(W_i \geq h) + o(h) = e^{-a_Sxh} + o(h) = 1 - a_Sxh + o(h)$
\item $P(S(t+h) = s+1 | S(t) = s, X(t) = x) = x(e^{-a_S(x-1)h} - e^{-a_Sxh}) + o(h) = a_Sxh + o(h)$
\item $P(S(t+h) = s+2 | S(t) = s, X(t) = x) = o(h)$
\end{itemize}
$o(h)$ pojawia się już po pierwszej równości ze względu na to, że przy dokładnym rozpisaniu prawdopodobienstw należałoby warunkować w którym momencie X(t) się zmieni (X jest zależny od S), jednak ta różnica jest $o(h)$, więc po prostu jest zawarta w tym.

zatem
\[P(X(t+h) = x+1 | X(t) = x) = \]
\[P(N(t+h) = n + 1 | N(t) = n) \cdot P(S(t+h) = s| S(t) = s, X(t)=x) = \]
\[(a_N h + o(h)) \cdot (1 - a_Sxh + o(h)) = \]
\[a_N h + o(h) \] 

\[P(X(t+h) = x-1 | X(t) = x) = \]
\[P(N(t+h) = n | N(t) = n) \cdot P(S(t+h) = s+1| S(t) = s, X(t)=x) = \]
\[(1 - a_N h + o(h)) \cdot (a_Sxh + o(h)) = \]
\[a_S xh + o(h) \] 

Czyli mamy \begin{itemize}
\item$P(X(t+h) = x+1 | X(t) = x) = a_N h + o(h)$
\item$P(X(t+h) = x-1 | X(t) = x) = a_S xh + o(h)$
\end{itemize}
wzory te razem z $Q(x,x) = 1-a_Nh - a_Sxh$ i zerami w pozostałych wierszach opisują intensywność przejść procesu Markowa


\end{document}