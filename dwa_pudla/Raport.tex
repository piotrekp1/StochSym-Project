\documentclass{article}
\usepackage{graphicx}
\usepackage{polski}
\usepackage[utf8]{inputenc}
\usepackage{graphicx}

\usepackage{subcaption}
\usepackage{algpseudocode}
\usepackage{amssymb}

\usepackage[a5paper]{geometry}
\graphicspath{ {./wykresy/} }
\begin{document}
\title{Model Chandrasekhara / Smoluchowskiego  - 2 pudła}
\author{Piotr Piękos}

\maketitle

W tym przypadku rozpatrujemy przypadek z dwoma pudełkami. 

Osoba "rodzi się" na podstawie procesu Poissona z częstotliwością $a_N$ trafiając do pudełka 1, spędza tam czas będący zmienną losową o rozkładzie wykładniczym z parametrem $a_{S_1}$. Następnie przechodzi do drugiego pudełka w którym spędza czas będący zmienną losową o rozkładzie wykładniczym z parametrem $a_{S_2}$
\section{Prawa Ewolucji}

Prawa ewolucji są naturalnym rozszerzem praw ewolucji z procesu dla jednego pudełka. W szczególności pierwsze pudełko musi operować na dokładnie tych samych zasadach co poprzednio.

\[P(X_1(t+h) = x_1 + 1, X_2(t+h)=x_2 | X_1(t) = x_1, X_2(t) = x_2) = a_N h + o(h)\]
\[P(X_1(t+h) = x_1 - 1, X_2(t+h)=x_2 + 1| X_1(t) = x_1, X_2(t) = x_2) = a_{S_1} xh + o(h)\]
\[P(X_1(t+h) = x_1, X_2(t+h)=x_2 - 1| X_1(t) = x_1, X_2(t) = x_2) = a_{S_2} xh + o(h)\]
\[P(X_1(t+h) = x_1, X_2(t+h)=x_2| X_1(t) = x_1, X_2(t) = x_2) = 1 - (a_{S_1}  + a_{S_2})xh + o(h)\]
dla każdego innego przejścia prawdopodobieństwo to $o(h)$

Przy naturalnych założeniach $x_1 > 0$ lub $x_2 > 0$ kiedy się zmniejsza.

\section{Algorytm}

Algorytm jest naturalnym rozszerzeniem algorytmu symulacji jednego pudełka



\begin{algorithmic}
\State Gen $N \sim \textit{Poiss}(a_Nt)$
\State for $i = 1$ to N do Gen $U_i \sim U(0, t)$, Gen $L1_i \sim \textit{Exp}(1/a_{S_1})$, Gen $L2_i \sim \textit{Exp}(1/a_{S_2})$
\State $(T_1, ..., T_n) =$ Sort$(U_1, ..., U_n)$ otrzymujemy proces Poissona, czasy narodzin
\State $(D1_1, ..., D1_n) = (T_1 + L1_1, ..., T_n + L1_n)$ - dodajemy niezalezne czasy zycia do czasow narodzin i mamy czasy przejscia do drugiego pudelka.
\State $(D2_1, ..., D2_n) = (D1_1 + L2_1, ..., D1_n + L2_n)$ - dodajemy niezalezne czasy zycia do czasow narodzin i mamy czasy smierci.
\end{algorithmic}  

Czyli od poprzedniego algorytmu różni się jedynie tym że losujemy dodatkowo czasy przebywania w drugim pudełku.

\end{document}